% Options for packages loaded elsewhere
\PassOptionsToPackage{unicode}{hyperref}
\PassOptionsToPackage{hyphens}{url}
\PassOptionsToPackage{dvipsnames,svgnames,x11names}{xcolor}
%
\documentclass[
]{agujournal2019}

\usepackage{amsmath,amssymb}
\usepackage{iftex}
\ifPDFTeX
  \usepackage[T1]{fontenc}
  \usepackage[utf8]{inputenc}
  \usepackage{textcomp} % provide euro and other symbols
\else % if luatex or xetex
  \usepackage{unicode-math}
  \defaultfontfeatures{Scale=MatchLowercase}
  \defaultfontfeatures[\rmfamily]{Ligatures=TeX,Scale=1}
\fi
\usepackage{lmodern}
\ifPDFTeX\else  
    % xetex/luatex font selection
\fi
% Use upquote if available, for straight quotes in verbatim environments
\IfFileExists{upquote.sty}{\usepackage{upquote}}{}
\IfFileExists{microtype.sty}{% use microtype if available
  \usepackage[]{microtype}
  \UseMicrotypeSet[protrusion]{basicmath} % disable protrusion for tt fonts
}{}
\makeatletter
\@ifundefined{KOMAClassName}{% if non-KOMA class
  \IfFileExists{parskip.sty}{%
    \usepackage{parskip}
  }{% else
    \setlength{\parindent}{0pt}
    \setlength{\parskip}{6pt plus 2pt minus 1pt}}
}{% if KOMA class
  \KOMAoptions{parskip=half}}
\makeatother
\usepackage{xcolor}
\setlength{\emergencystretch}{3em} % prevent overfull lines
\setcounter{secnumdepth}{5}
% Make \paragraph and \subparagraph free-standing
\ifx\paragraph\undefined\else
  \let\oldparagraph\paragraph
  \renewcommand{\paragraph}[1]{\oldparagraph{#1}\mbox{}}
\fi
\ifx\subparagraph\undefined\else
  \let\oldsubparagraph\subparagraph
  \renewcommand{\subparagraph}[1]{\oldsubparagraph{#1}\mbox{}}
\fi


\providecommand{\tightlist}{%
  \setlength{\itemsep}{0pt}\setlength{\parskip}{0pt}}\usepackage{longtable,booktabs,array}
\usepackage{calc} % for calculating minipage widths
% Correct order of tables after \paragraph or \subparagraph
\usepackage{etoolbox}
\makeatletter
\patchcmd\longtable{\par}{\if@noskipsec\mbox{}\fi\par}{}{}
\makeatother
% Allow footnotes in longtable head/foot
\IfFileExists{footnotehyper.sty}{\usepackage{footnotehyper}}{\usepackage{footnote}}
\makesavenoteenv{longtable}
\usepackage{graphicx}
\makeatletter
\def\maxwidth{\ifdim\Gin@nat@width>\linewidth\linewidth\else\Gin@nat@width\fi}
\def\maxheight{\ifdim\Gin@nat@height>\textheight\textheight\else\Gin@nat@height\fi}
\makeatother
% Scale images if necessary, so that they will not overflow the page
% margins by default, and it is still possible to overwrite the defaults
% using explicit options in \includegraphics[width, height, ...]{}
\setkeys{Gin}{width=\maxwidth,height=\maxheight,keepaspectratio}
% Set default figure placement to htbp
\makeatletter
\def\fps@figure{htbp}
\makeatother
% definitions for citeproc citations
\NewDocumentCommand\citeproctext{}{}
\NewDocumentCommand\citeproc{mm}{%
  \begingroup\def\citeproctext{#2}\cite{#1}\endgroup}
\makeatletter
 % allow citations to break across lines
 \let\@cite@ofmt\@firstofone
 % avoid brackets around text for \cite:
 \def\@biblabel#1{}
 \def\@cite#1#2{{#1\if@tempswa , #2\fi}}
\makeatother
\newlength{\cslhangindent}
\setlength{\cslhangindent}{1.5em}
\newlength{\csllabelwidth}
\setlength{\csllabelwidth}{3em}
\newenvironment{CSLReferences}[2] % #1 hanging-indent, #2 entry-spacing
 {\begin{list}{}{%
  \setlength{\itemindent}{0pt}
  \setlength{\leftmargin}{0pt}
  \setlength{\parsep}{0pt}
  % turn on hanging indent if param 1 is 1
  \ifodd #1
   \setlength{\leftmargin}{\cslhangindent}
   \setlength{\itemindent}{-1\cslhangindent}
  \fi
  % set entry spacing
  \setlength{\itemsep}{#2\baselineskip}}}
 {\end{list}}
\usepackage{calc}
\newcommand{\CSLBlock}[1]{\hfill\break#1\hfill\break}
\newcommand{\CSLLeftMargin}[1]{\parbox[t]{\csllabelwidth}{\strut#1\strut}}
\newcommand{\CSLRightInline}[1]{\parbox[t]{\linewidth - \csllabelwidth}{\strut#1\strut}}
\newcommand{\CSLIndent}[1]{\hspace{\cslhangindent}#1}

\usepackage{url} %this package should fix any errors with URLs in refs.
\usepackage{lineno}
\usepackage[inline]{trackchanges} %for better track changes. finalnew option will compile document with changes incorporated.
\usepackage{soul}
\linenumbers
\makeatletter
\@ifpackageloaded{caption}{}{\usepackage{caption}}
\AtBeginDocument{%
\ifdefined\contentsname
  \renewcommand*\contentsname{Table of contents}
\else
  \newcommand\contentsname{Table of contents}
\fi
\ifdefined\listfigurename
  \renewcommand*\listfigurename{List of Figures}
\else
  \newcommand\listfigurename{List of Figures}
\fi
\ifdefined\listtablename
  \renewcommand*\listtablename{List of Tables}
\else
  \newcommand\listtablename{List of Tables}
\fi
\ifdefined\figurename
  \renewcommand*\figurename{Figure}
\else
  \newcommand\figurename{Figure}
\fi
\ifdefined\tablename
  \renewcommand*\tablename{Table}
\else
  \newcommand\tablename{Table}
\fi
}
\@ifpackageloaded{float}{}{\usepackage{float}}
\floatstyle{ruled}
\@ifundefined{c@chapter}{\newfloat{codelisting}{h}{lop}}{\newfloat{codelisting}{h}{lop}[chapter]}
\floatname{codelisting}{Listing}
\newcommand*\listoflistings{\listof{codelisting}{List of Listings}}
\makeatother
\makeatletter
\@ifpackageloaded{caption}{}{\usepackage{caption}}
\@ifpackageloaded{subcaption}{}{\usepackage{subcaption}}
\makeatother
\makeatletter
\makeatother
\ifLuaTeX
  \usepackage{selnolig}  % disable illegal ligatures
\fi
\IfFileExists{bookmark.sty}{\usepackage{bookmark}}{\usepackage{hyperref}}
\IfFileExists{xurl.sty}{\usepackage{xurl}}{} % add URL line breaks if available
\urlstyle{same} % disable monospaced font for URLs
\hypersetup{
  pdftitle={La Palma Earthquakes},
  pdfauthor={Steve Purves; Rowan Cockett},
  pdfkeywords={La Palma, Earthquakes},
  colorlinks=true,
  linkcolor={blue},
  filecolor={Maroon},
  citecolor={Blue},
  urlcolor={Blue},
  pdfcreator={LaTeX via pandoc}}

\journalname{Notebooks Now!}

\draftfalse

\begin{document}
\title{La Palma Earthquakes}

\authors{Steve Purves\affil{1}, Rowan Cockett\affil{1}}
\affiliation{1}{Curvenote, }
\correspondingauthor{Steve Purves}{steve@curvenote.com}

\begin{keypoints}
\item You may specify 1 to 3 keypoints for this PDF template \item These
keypoints are complete sentences and less than or equal to 140
characters \item They are specific to this PDF template, so they will
not appear in other exports 
\end{keypoints}

\begin{abstract}
In September 2021, a significant jump in seismic activity on the island
of La Palma (Canary Islands, Spain) signaled the start of a volcanic
crisis that still continues at the time of writing. Earthquake data is
continually collected and published by the Instituto Geográphico
Nacional (IGN). We have created an accessible dataset from this and
completed preliminary data analysis which shows seismicity originating
at two distinct depths, consistent with the model of a two reservoir
system feeding the currently very active volcano.
\end{abstract}



\subsection{Introduction}\label{introduction}

\begin{quote}
The content of your notebook may be broken into any number of markdown
or code cells. Markdown cells use Quarto markdown. Quarto markdown
supports an extended version of the basic Markdown syntax originally
created by John Gruber, which adds support for many common document
elements including citations, figures, tables, admonitions, and more.
Quarto markdown also supports the use of LaTeX for mathematical
equations, advanced layout control, as well as other advanced
formatting.
\end{quote}

La Palma is one of the west most islands in the Volcanic Archipelago of
the Canary Islands, a Spanish territory situated is the Atlantic Ocean
where at their closest point are 100km from the African coast
Figure~\ref{fig-map}. The island is one of the youngest, remains active
and is still in the island forming stage.

\begin{quote}
Figures may be added to your notebook using
\href{https://quarto.org/docs/authoring/figures.html}{markdown images or
specifial markdown elements} (`fenced divs'). They may refer to images
saved in your \texttt{images/} folder (or other folders), images from
the web, or generated directly using code cells. You may embed figures
produced in other notebooks using the
\href{https://quarto.org/docs/authoring/notebook-embed.html}{embed
shortcode} (this embed figures, tables, or any other content from
Jupyter Notebooks). Refer to figures by their label
(e.g.~\texttt{@fig-map}).
\end{quote}

\begin{figure}

\centering{

\includegraphics[width=1\textwidth,height=\textheight]{images/la-palma-map.png}

}

\caption{\label{fig-map}Map of La Palma in the Canary Islands. Image
credit
\href{https://commons.wikimedia.org/w/index.php?curid=76638603}{NordNordWest}}

\end{figure}%

La Palma has been constructed by various phases of volcanism, the most
recent and currently active being the \emph{Cumbre Vieja} volcano, a
north-south volcanic ridge that constitutes the southern half of the
island.

\subsubsection{Eruption History}\label{eruption-history}

A number of eruptions were recorded since the colonization of the
islands by Europeans in the late 1400s, these are summarised in
Table~\ref{tbl-history}.

\begin{quote}
Quarto supports a number of ways to create tables using both standard
markdown tables (pipe tables) and more complex markdown tables using a
grid style syntax (grid tables). In addition, Quarto provides the
ability to control column width, caption position, create subtables, and
more. See \href{https://quarto.org/docs/authoring/tables.html}{Quarto's
table document} to learn more. Refer to tables in the text by their
label (e.g.~\texttt{@tbl-history}).
\end{quote}

\begin{longtable}[]{@{}ll@{}}
\caption{Recent historic eruptions on La
Palma}\label{tbl-history}\tabularnewline
\toprule\noalign{}
Name & Year \\
\midrule\noalign{}
\endfirsthead
\toprule\noalign{}
Name & Year \\
\midrule\noalign{}
\endhead
\bottomrule\noalign{}
\endlastfoot
Current & 2021 \\
Teneguía & 1971 \\
Nambroque & 1949 \\
El Charco & 1712 \\
Volcán San Antonio & 1677 \\
Volcán San Martin & 1646 \\
Tajuya near El Paso & 1585 \\
Montaña Quemada & 1492 \\
\end{longtable}

This equates to an eruption on average every 79 years up until the 1971
event. The probability of a future eruption can be modeled by a Poisson
distribution Equation~\ref{eq-poisson}.

\begin{quote}
Numbered equations may be defined using `dollar math' by placing
equations between matching pairs of dollar signs. Learn more about
Quarto's equation here:
\url{https://quarto.org/docs/authoring/cross-references.html\#equations}.
Refer to equations in the text by their label
(e.g.~\texttt{@eq-poisson}).
\end{quote}

\begin{equation}\phantomsection\label{eq-poisson}{
p(x)=\frac{e^{-\lambda} \lambda^{x}}{x !}
}\end{equation}

Where \(\lambda\) is the number of eruptions per year,
\(\lambda=\frac{1}{79}\) in this case. The probability of a future
eruption in the next \(t\) years can be calculated by:

\begin{equation}\phantomsection\label{eq-probability}{
p_e = 1-\mathrm{e}^{-t \lambda}
}\end{equation}

So following the 1971 eruption the probability of an eruption in the
following 50 years --- the period ending this year --- was 0.469. After
the event, the number of eruptions per year moves to
\(\lambda=\frac{1}{75}\) and the probability of a further eruption
within the next 50 years (2022-2071) rises to 0.487 and in the next 100
years, this rises again to 0.736.

\subsubsection{Magma Reservoirs}\label{magma-reservoirs}

\begin{quote}
You may provide the bibliography directly as a bibtex, biblatex, CSL
JSON, or CSL YAML file (defined in the document front matter or
\texttt{\_quarto.yml} project file) then embed the citation by citation
key in your text using the \texttt{{[}@cite{]}} or \texttt{@cite} for
parenthetical or textual citations, respectively. The following
paragraph provides an example of this. Quarto's
\href{https://quarto.org/docs/authoring/footnotes-and-citations.html}{documentation
on citations} provides more details on working with bibliographies and
citations.
\end{quote}

Studies of the magma systems feeding the volcano, such as Marrero et al.
(2019) has proposed that there are two main magma reservoirs feeding the
Cumbre Vieja volcano; one in the mantle (30-40km depth) which charges
and in turn feeds a shallower crustal reservoir (10-20km depth).

\begin{figure}

\centering{

\includegraphics[width=1\textwidth,height=\textheight]{images/reservoirs.png}

}

\caption{\label{fig-reservoirs}Proposed model from Marrero et al}

\end{figure}%

In this paper, we look at recent seismicity data to see if we can see
evidence of such a system action, see Figure~\ref{fig-reservoirs}.

\section{Dataset}\label{dataset}

\begin{quote}
All data used in the notebook should be present in the \texttt{data/}
folder so notebooks may be executed in place with no additional input.
\end{quote}

The earthquake dataset used in our analysis was generated from the
\href{https://www.ign.es/web/resources/volcanologia/tproximos/canarias.html}{IGN
web portal} this is public data released under a permissive license.
Data recorded using the network of Seismic Monitoring Stations on the
island. A web scraping script was developed to pull data into a
machine-readable form for analysis. That code tool
\href{https://github.com/stevejpurves/ign-earthquake-data}{is available
on GitHub} along with a copy of recently updated data.

\subsection{Main Timeline Figure}\label{main-timeline-figure}

\begin{quote}
Code cells may be seamlessly interleaved with markdown cells. There are
a variety of execution options to control the behavior of code cells -
learn more in
\href{https://quarto.org/docs/computations/execution-options.html}{Quarto's
documentation on execution options}.
\end{quote}

\textsubscript{Source:
\href{https://Notebooks-Now.github.io/submission-quarto-lite-r/index.qmd.html}{Article
Notebook}}

\subsection{Visualising Long term earthquake
data}\label{visualising-long-term-earthquake-data}

Data taken directly from the IGN Catalog

\begin{quote}
Supported cell outputs below include \texttt{pandas} dataframe, raw text
output, \texttt{matplotlib} plot, and \texttt{seaborn} plot.
\end{quote}

\textsubscript{Source:
\href{https://Notebooks-Now.github.io/submission-quarto-lite-r/index.qmd.html}{Article
Notebook}}

\textsubscript{Source:
\href{https://Notebooks-Now.github.io/submission-quarto-lite-r/index.qmd.html}{Article
Notebook}}

\textsubscript{Source:
\href{https://Notebooks-Now.github.io/submission-quarto-lite-r/index.qmd.html}{Article
Notebook}}

\textsubscript{Source:
\href{https://Notebooks-Now.github.io/submission-quarto-lite-r/index.qmd.html}{Article
Notebook}}

\begin{figure}[H]

\centering{

\includegraphics{index_files/figure-pdf/plot-timeline-1.pdf}

}

\end{figure}%

\textsubscript{Source:
\href{https://Notebooks-Now.github.io/submission-quarto-lite-r/index.qmd.html}{Article
Notebook}}

\subsection{Cumulative Distribution
Plots}\label{cumulative-distribution-plots}

\begin{figure}[H]

\centering{

\includegraphics{index_files/figure-pdf/plot-dists-1.pdf}

}

\end{figure}%

\textsubscript{Source:
\href{https://Notebooks-Now.github.io/submission-quarto-lite-r/index.qmd.html}{Article
Notebook}}

\section{Results}\label{results}

The dataset was loaded into this Jupyter notebook and filtered down to
La Palma events only. This results in 5465 data points which we then
visualized to understand their distributions spatially, by depth, by
magnitude and in time.

From our analysis above, we can see 3 different systems in play.

Firstly, the shallow earthquake swarm leading up to the eruption on 19th
September, related to significant surface deformation and shallow magma
intrusion.

After the eruption, continuous shallow seismicity started at 10-15km
corresponding to magma movement in the crustal reservoir.

Subsequently, high magnitude events begin occurring at 30-40km depths
corresponding to changes in the mantle reservoir. These are also
continuous but occur with a lower frequency than in the crustal
reservoir.

\section{Conclusions}\label{conclusions}

From the analysis of the earthquake data collected and published by IGN
for the period of 11 September through to 9 November 2021. Visualization
of the earthquake events at different depths appears to confirm the
presence of both mantle and crustal reservoirs as proposed by Marrero et
al. (2019).

\begin{quote}
Data availability statement should be specified in a separate cell with
metadata \texttt{"part":\ "availability"}, similar to the abstract.

AGU requires an Availability Statement for the underlying data needed to
understand, evaluate, and build upon the reported research at the time
of peer review and publication.
\end{quote}

A web scraping script was developed to pull data into a machine-readable
form for analysis. That code tool
\href{https://github.com/stevejpurves/ign-earthquake-data}{is available
on GitHub} along with a copy of recently updated data.

\subsection*{References}\label{references}
\addcontentsline{toc}{subsection}{References}

\vspace{1em}

\phantomsection\label{refs}
\begin{CSLReferences}{1}{0}
\bibitem[\citeproctext]{ref-marrero2019}
Marrero, J., García, A., Berrocoso, M., Llinares, Á., Rodríguez-Losada,
A., \& Ortiz, R. (2019). Strategies for the development of volcanic
hazard maps in monogenetic volcanic fields: The example of {La} {Palma}
({Canary} {Islands}). \emph{Journal of Applied Volcanology}, \emph{8}.
\url{https://doi.org/10.1186/s13617-019-0085-5}

\end{CSLReferences}



\end{document}
